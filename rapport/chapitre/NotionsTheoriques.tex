\documentclass[../rapportdestage.tex]{subfiles}

 
\begin{document}
 


\section{Notions théoriques}
	\subsection{Java Entreprise Edition (J2EE)}
Le langage Java, développé par Sun, est un langage orienté objet, en effet il possède un
mécanisme qui permet de décrire les caractéristiques d'un objets de façon unique et de
pouvoir lui faire subir des opérations.

Le langage Java n'est pas interprété mais les fichiers java, appelés portant l'extension
.java, sont compilés en byte code, fichier .class, puis lus par ce que l'on appelle une
machine virtuelle Java (JVM).

 Le langage est donc indépendant de chaque machine, on
parle de langage portable.

Pour écrire du Java, il suffit d'installer un JRE(Java Runtime Environement) ou un
JDK(Java Developpement Kit) qui comprend un JRE et d'autre outils, une JVM et un simple éditeur de texte.

JEE, qui peut être considéré comme une extension de Java, est un ensemble de
spécifications destinées aux applications d'entreprises. Ce langage permet la création
d'applications performantes et robustes.

JEE s'appuie sur le modèle Modèle Vue Contrôleur (MVC)		
	
		
	\subsection{Architecture (MVC)}
Il est primordiale à la conception de tout système informatique de choisir le modèle d’architecture
qui lui sera adéquat pouvant assurer un bon fonctionnement, des meilleurs performances ainsi que
la réutilisation et l’interconnexion fiable de ce système avec d’autres. 

Nous avons choisi le modèle
MVC « Model View Controller » comme notre modèle architectural qui est adapté particulièrement
à la création des applications Web. Un avantage apporté par ce modèle est la clarté de l’architecture
qu’il impose.

Cela simplifie la tâche du développeur qui tenterait d’effectuer une maintenance ou une
amélioration sur le projet et la diminution de la complexité lors de la conception.
Cette méthode différentie trois parties dans une application web qui sont :
\begin{itemize}
\item vues : Ensembles de fichiers responsables de la génération du HTML, en bref, c’est l’interface Homme-Machine de l’application 
\item modèles : Présentent des fonctions qui aident à récupérer, insérer et mettre à jour des informations de la base de données
\item  contrôleurs : Des fichiers qui agissent sur les données et effectue la synchronisation entre le modèle et la vue.
\end{itemize}

	\subsection{Architecture 3-tiers}
	Dans l'architecture 3-tiers (tiers signifiant étages en
anglais) il existe un niveau intermédiaire, c'est-à-dire que l'on a généralement une architecture partagée entre:

1. Le client: le demandeur de ressources

2. Le serveur d'application (appelé aussi middleware): le serveur chargé de fournir la ressource mais faisant appel à un autre serveur

3. Le serveur secondaire (généralement un serveur de base de données), fournissant un
service au premier serveur.
	
	\end{document}